\RequirePackage[l2tabu, orthodox]{nag}
\RequirePackage{silence}
\WarningsOff*

\documentclass[%
    pdftex,
    oneside,			% Einseitiger Druck.
    12pt,				% Schriftgroesse
    parskip=half,		% Halbe Zeile Abstand zwischen Absätzen.
    %topmargin = 10pt,	% Abstand Seitenrand (Std:1in) zu Kopfzeile [laut log: unused]
    headheight = 33pt,	% Höhe der Kopfzeile
    %headsep = 30pt,	% Abstand zwischen Kopfzeile und Text Body  [laut log: unused]
    headsepline,		% Linie nach Kopfzeile.
    footsepline,		% Linie vor Fusszeile.
    %footheight = 16pt,	% Höhe der Fusszeile
    abstracton,		% Abstract Überschriften
    DIV=calc,		% Satzspiegel berechnen
    BCOR=8mm,		% Bindekorrektur links: 8mm
    headinclude=false,	% Kopfzeile nicht in den Satzspiegel einbeziehen
    footinclude=false,	% Fußzeile nicht in den Satzspiegel einbeziehen
    listof=totoc,		% Abbildungs-/ Tabellenverzeichnis im Inhaltsverzeichnis darstellen
    toc=bibliography,	% Literaturverzeichnis im Inhaltsverzeichnis darstellen
]{scrreprt}	% Koma-Script report-Klasse, fuer laengere Bachelorarbeiten alternativ auch: scrbook

%%%%%%% Package Includes %%%%%%%
\usepackage{xstring}
\usepackage{ifpdf}
\usepackage{ifluatex}
\usepackage{lastpage}
\usepackage{fancyhdr} % Causing Warnings
\usepackage{pdfpages} % pdf-Seiten einbinden
\usepackage[utf8]{inputenc}
\usepackage[T1]{fontenc}
\usepackage{tikz}
\usepackage{xcolor}
\usepackage[margin=2.5cm,foot=1cm,top=3cm,bottom=3cm]{geometry}	% Seitenränder und Abstände
\usepackage[activate]{microtype} %Zeilenumbruch und mehr
\usepackage{varwidth}
\usepackage[onehalfspacing]{setspace}
\usepackage{makeidx}
\usepackage[autostyle=true,german=quotes]{csquotes}
\usepackage{longtable}
\usepackage{enumitem}	% mehr Optionen bei Aufzählungen
\usepackage{graphicx}
\usepackage{xcolor} 	% für HTML-Notation
\usepackage{float}
\usepackage{array}
\usepackage{calc}		% zum Rechnen (Bildtabelle in Deckblatt)
\usepackage{wrapfig}
\usepackage{pgffor} % für automatische Kapiteldateieinbindung
\usepackage[perpage, hang, multiple, stable]{footmisc} % Fussnoten
\usepackage{acronym}
\usepackage[absolute]{textpos}
\usepackage{scrhack} % in Kombination mit listings-Package kommt es zu Warnings, dieses Paket verhindert die Warnings! Ggf. auskommentieren und die Warnings akzeptieren falls Verzeichnisse nicht so dargestellt werden wie gewünscht
\usepackage{listings} % Code-Listings
\usepackage{color, colortbl}  %Für Highlighten der Tabellenzeilen
\usepackage{amsmath}% http://ctan.org/pkg/amsmath
\usepackage[ngerman]{babel}
\usepackage[font=footnotesize,lofdepth,lotdepth]{subfig}
% eine Kommentarumgebung "k" (Handhabe mit \begin{k}<Kommentartext>\end{k},
% Kommentare werden rot gedruckt). Wird \% vor excludecomment{k} entfernt,
% werden keine Kommentare mehr gedruckt.
\usepackage{comment}

\newcommand{\einstellung}[1]{%
    \expandafter\newcommand\csname #1\endcsname{}
    \expandafter\newcommand\csname setze#1\endcsname[1]{\expandafter\renewcommand\csname#1\endcsname{##1}}
} %Einstellungscommand

\newcommand{\langstr}[1]{\einstellung{lang#1}} % Sprache aus Einstellungen laden

\newcommand{\citem}[1]{\item[\texttt{#1}]} % Code-Item für description-Liste

\newcommand{\todo}[1]{\textit{\textcolor{red}{TODO: #1}}} % Todo-Item

\newcommand{\ladefarben}{%
	\definecolor{LinkColor}{HTML}{00007A}
	\definecolor{ListingBackground}{HTML}{FCF7DE}
} % Farben (Angabe in HTML-Notation mit großen Buchstaben)

%% Programmiersprachen Highlighting (Listings)
\newcommand{\listingsettings}{%
	\lstset{%
		language=C++,			% Standardsprache des Quellcodes
		%numbers=left,			% Zeilennummern links
		%stepnumber=1,			% Jede Zeile nummerieren.
		%numbersep=5pt,			% 5pt Abstand zum Quellcode
		%numberstyle=\tiny,		% Zeichengrösse 'tiny' für die Nummern.
		breaklines=true,		% Zeilen umbrechen wenn notwendig.
		breakautoindent=true,	% Nach dem Zeilenumbruch Zeile einrücken.
		postbreak=\space,		% Bei Leerzeichen umbrechen.
		tabsize=2,				% Tabulatorgrösse 2
		basicstyle=\ttfamily\footnotesize, % Nichtproportionale Schrift, klein für den Quellcode
		showspaces=false,		% Leerzeichen nicht anzeigen.
		showstringspaces=false,	% Leerzeichen auch in Strings ('') nicht anzeigen.
		extendedchars=true,		% Alle Zeichen vom Latin1 Zeichensatz anzeigen.
		captionpos=b,			% sets the caption-position to bottom
		%backgroundcolor=\color{ListingBackground}, % Hintergrundfarbe des Quellcodes setzen.
		xleftmargin=0pt,		% Rand links
		xrightmargin=0pt,		% Rand rechts
		frame=single,			% Rahmen an
		frameround=ffff,
		rulecolor=\color{darkgray},	% Rahmenfarbe
		%fillcolor=\color{ListingBackground},
		keywordstyle=\color[rgb]{0.133,0.133,0.6},
		commentstyle=\color[rgb]{0.133,0.545,0.133},
		stringstyle=\color[rgb]{0.627,0.126,0.941},
    aboveskip=1.5em,
	}
}

%%%%%%%%%%%%%%%%%%%%%%%%%%%%% Kopf-/Fußzeilenwechsel %%%%%%%%%%%%%%%%%%%%%%%%%%%
\setlength{\headheight}{40pt}

\newcommand{\setpagestyle}[1]{%
    \fancypagestyle{plain}{%
    \fancyhf{}
    \fancyhead[L]{\autor}
    \fancyhead[R]{
    \rightmark
    }
    \fancyfoot[L]{
        \noindent{\tiny \langfussz\\
            \begin{tabular*}{16cm}{@{\extracolsep{\fill}}l>{\raggedleft}p{8cm}}
                {\tiny \langstand: \datumAbgabe} & 
                {\tiny \langseite\ \thepage\ \langseitevon\ \pageref*{#1}\vspace{1cm}}\tabularnewline
            \end{tabular*}
        }
    }
    }
    \pagestyle{plain}
}

\newcommand{\setpagestylehead}{%
    \setpagestyle{endOfRomanNumbering}
    \pagenumbering{roman}
}

\newcommand{\setpagestylecontent}{
    \setpagestyle{endOfArabicNumbering}
    \pagenumbering{arabic}
}

\newcommand{\setpagestylefoot}{
    \setpagestyle{LastPage}
    \pagenumbering{Alph}
}

% Flag für die Selbstständigkeitserklärung, Default: true
\newif\ifselbsterkl
\selbsterklfalse

% Flag für das Wasserzeichen auf dem Deckblatt, default: false
\newif\ifwatermark
\watermarkfalse

% Flag für roten Vertraulichkeitspunkt, default: false
\newif\ifreddot
\reddotfalse

% Flag für gelben Vertraulichkeitspunkt, default: false
\newif\ifyellowdot
\yellowdotfalse

% Flag für das Unterschriftenblatt, default: false
\newif\ifunterschriftenblatt
\unterschriftenblattfalse

% Flag für Einfügen der Seitenzahl bei Verweis auf Kapitel/Abschnitt, default: false
\newif\ifrefWithPages
\refWithPagesfalse

% Flag für Einfügen der Abstracts in deutsch und englisch, default: false
\newif\ifbothabstracts
\bothabstractsfalse

% Flag für Einfügen des Abkürzungsverzeichnis
\newif\ifabkverz
\abkverzfalse

% Flag für Einfügen des Abbildungsverzeichnisses
\newif\ifabbverz
\abbverzfalse

% Flag für Einfügen des Formelverzeichnisses
\newif\ifformelverz
\formelverzfalse

% Flag für Einfügen des Formelgroessenverzeichnisses
\newif\ifformelgroeverz
\formelgroeverzfalse 

% Flag für Einfügen des Listingsverzeichnisses
\newif\iflistverz
\listverzfalse

% Flag für Einfügen des Tabellenverzeichnisses
\newif\iftableverz
\tableverzfalse

% Flag für Einfügen des Sperrvermerks
\newif\ifsperrvermerk
\sperrvermerkfalse

% Flag für Einfügen des Abstracts
\newif\ifabstract
\abstractfalse

% Flag für Anhang
\newif\ifappendix
\appendixfalse

% Flag für Literaturverzeichnis
\newif\ifliteratur
\literaturfalse

%Flag für untereiltest Literaturverzeichnis
\newif\ifliteraturteilen
\literaturteilenfalse

% Flag für Glossar
\newif\ifglossar
\glossarfalse

% Flag für Inhaltsverzeichnis
\newif\ifinhalt
\inhaltfalse

% Flag für Reviewer
\newif\ifreviewer
\reviewerfalse

\input{ads/settings_list.tex} % verfügbare Einstellungen
% !TeX root = dokumentation.tex
%%%%%%%%%%%%%%%%%%%%%%%%%%%%%%%%%%%%%%%%%%%%%%%%%%%%%%%%%%%%%%%%%%%%%%%%%%%%%%%
%                                   Einstellungen
%
% Hier können alle relevanten Einstellungen für diese Arbeit gesetzt werden.
% Dazu gehören Angaben u.a. über den Autor sowie Formatierungen.
%
%
%%%%%%%%%%%%%%%%%%%%%%%%%%%%%%%%%%%%%%%%%%%%%%%%%%%%%%%%%%%%%%%%%%%%%%%%%%%%%%%


%%%%%%%%%%%%%%%%%%%%%%%%%%%%%%%%%%%% Sprache %%%%%%%%%%%%%%%%%%%%%%%%%%%%%%%%%%%
%% Aktuell sind Deutsch und Englisch unterstützt.
%% Es werden nicht nur alle vom Dokument erzeugten Texte in
%% der entsprechenden Sprache angezeigt, sondern auch weitere
%% Aspekte angepasst, wie z.B. die Anführungszeichen und
%% Datumsformate.
\setzesprache{de} % de oder en
%%%%%%%%%%%%%%%%%%%%%%%%%%%%%%%%%%%%%%%%%%%%%%%%%%%%%%%%%%%%%%%%%%%%%%%%%%%%%%%%

%%%%%%%%%%%%%%%%%%%%%%%%%%%%%%%%%%% Angaben  %%%%%%%%%%%%%%%%%%%%%%%%%%%%%%%%%%%
%% Die meisten der folgenden Daten werden auf dem
%% Deckblatt angezeigt, einige auch im weiteren Verlauf
%% des Dokuments.
\setzemartrikelnr{-remove-}
\setzekurs{-remove-}
\setzetitel{-remove-}
\setzedatumAnfang{-remove-}
\setzedatumAbgabe{-remove-}
\setzefirma{-remove-}
\setzefirmenort{-remove-}
\setzeabgabeort{-remove-}
\setzeabschluss{-remove-}
\setzestudiengang{-remove-}
\setzedhbw{-remove-}
\setzebetreuer{-remove-}
\setzegutachter{-remove-}
\setzezeitraum{-remove-}
\setzearbeit{-remove-}
\setzeautor{-remove-}
\setzesemester{-remove-}
\setzestudienrichtung{-remove-}
\setzejahrgang{-remove-}
\setzeabteilung{-remove-}
\setzestandort{-remove-}

\inhalttrue                 % auskommentieren oder ändern zu \inhaltfalse, falls kein Inhaltsverzeichnis eingefügt werden soll
% \unterschriftenblatttrue    % auskommentieren oder ändern zu \unterschriftenblattfalse, falls kein Unterschriftenblatt eingefügt werden soll
% \selbsterkltrue             % auskommentieren oder ändern zu \selbsterklfalse, wenn keine Selbstständigkeitserklärung benötigt wird
% \sperrvermerktrue           % auskommentieren oder ändern zu \sperrvermerkfalse, wenn kein Sperrvermerk benötigt wird
\abkverztrue                % auskommentieren oder ändern zu \abkverzfalse, wenn kein Abkürzungsverzeichnis benötigt wird
% \abbverztrue                % auskommentieren oder ändern zu \abbverzfalse, wenn kein Abbildungsverzeichnis benötigt wird
% \tableverztrue             % auskommentieren oder ändern zu \tableverzfalse, wenn kein Tabellenverzeichnis benötigt wird
% \listverztrue               % auskommentieren oder ändern zu \listverzfalse, wenn kein Listingsverzeichnis benötigt wird
% \formelverztrue             % auskommentieren oder ändern zu \formelverzfalse, wenn kein Formelverzeichnis benötigt wird
% \formelgroeverztrue		% auskommentieren oder ändern zu \formelgroeverzfalse, wenn kein Formelgrößenverzeichnis benötigt wird
\abstracttrue               % auskommentieren oder ändern zu \abstractfalse, wenn kein Abstract gewünscht ist
\bothabstractstrue          % auskommentieren oder ändern zu \bothabstractsfalse, wenn nur der Abstract in der Hauptsprache eingefügt werden soll
\appendixtrue               % auskommentieren oder ändern zu \appendixfalse, wenn kein Anhang gewünscht ist
\literaturtrue              % auskommentieren oder ändern zu \literaturfalse, wenn kein Literaturverzeichnis gewünscht ist (\appendixtrue muss gesetzt sein!)
% \literaturteilentrue        % ausommmentieren oder ändern zu \literaturteilenfalse, wenn Literaturverzeichnis nicht geteilt werden soll (\literaturtrue muss gesetzt sein!)
                            % ändern der Teilung in ads/appendix
% \glossartrue                % auskommentieren oder ändern zu \glossarfalse, wenn kein Glossar gewünscht ist (\appendixtrue muss gesetzt sein!)
% \watermarktrue             % auskommentieren oder ändern zu \watermarktrue, wenn Wasserzeichen auf dem Titelblatt eingefügt werden soll

\refWithPagesfalse          % ändern zu \refWithPagestrue, wenn die Seitenzahl bei Verweisen auf Kapitel engefügt werden sollen

%\reviewertrue				% auskommentiren oder ändern zu \reviewerfalse wenn kein Gutachter gesetzt werden muss

% Angabe des roten/gelben/grünen Punktes auf dem Titelblatt zur Kennzeichnung der Vertraulichkeitsstufe.
% Mögliche Angaben sind \yellowdottrue, \reddottrue. Werden beide angegeben, wird der rote Punkt gezeichnet.
% Wird keines der Kommandos angegeben, wird der grüner Punkt gezeichnet
% \reddottrue
% \yellowdottrue

%%%%%%%%%%%%%%%%%%%%%%%%%%%%%%%%%%%%%%%%%%%%%%%%%%%%%%%%%%%%%%%%%%%%%%%%%%%%%%%%

%%%%%%%%%%%%%%%%%%%%%%%%%%%% Literaturverzeichnis %%%%%%%%%%%%%%%%%%%%%%%%%%%%%%
%% Bei Fehlern während der Verarbeitung bitte in ads/header.tex bei der
%% Einbindung des Pakets biblatex (ungefähr ab Zeile 110,
%% einmal für jede Sprache), biber in bibtex ändern.
\newcommand{\ladeliteratur}{%
    \addbibresource{bibliographie.bib}
}

%% Zitierstil
%% siehe: http://ctan.mirrorcatalogs.com/macros/latex/contrib/biblatex/doc/biblatex.pdf (3.3.1 Citation Styles)
%% mögliche Werte z.B numeric-comp, alphabetic, authoryear
\setzezitierstil{ieee}
%%%%%%%%%%%%%%%%%%%%%%%%%%%%%%%%%%%%%%%%%%%%%%%%%%%%%%%%%%%%%%%%%%%%%%%%%%%%%%%%

%%%%%%%%%%%%%%%%%%%%%%%%%%%%%%%%% Layout %%%%%%%%%%%%%%%%%%%%%%%%%%%%%%%%%%%%%%%
%% Verschiedene Schriftarten
% laut nag Warnung: palatino obsolete, use mathpazo, helvet (option scaled=.95), courier instead
\setzeschriftart{lmodern} % palatino oder goudysans, lmodern, libertine

%% Abstand vor Kapitelüberschriften zum oberen Seitenrand
\setzekapitelabstand{20pt}

%% Spaltenabstand
\setzespaltenabstand{10pt}
%%Zeilenabstand innerhalb einer Tabelle
\setzezeilenabstand{1.5}
%%%%%%%%%%%%%%%%%%%%%%%%%%%%%%%%%%%%%%%%%%%%%%%%%%%%%%%%%%%%%%%%%%%%%%%%%%%%%%%% % lese Einstellungen

\input{lang/strings} % verfügbare Strings
\input{lang/\sprache} % Übersetzung einlesen

% Einstellung der Sprache des Paketes Babel und der Verzeichnisüberschriften

\PassOptionsToPackage{english, ngerman}{babel}

\iflang{de}{
    \usepackage{babel}
    \selectlanguage{ngerman}
}
\iflang{en}{
    \usepackage{babel}
    \selectlanguage{english}
}
%%%%%% Configuration %%%%%

%% Anwenden der Einstellungen
\usepackage{\schriftart}
\ladefarben

% Titel, Autor und Datum
\title{\titel}
\author{\autor}
\date{\datum}

% PDF Einstellungen
\usepackage[%
    pdftitle={\titel},
    pdfauthor={\autor},
    pdfsubject={\arbeit},
    pdfcreator={pdflatex, LaTeX with KOMA-Script},
    pdfpagemode=UseOutlines, 		% Beim Oeffnen Inhaltsverzeichnis anzeigen
    pdfdisplaydoctitle=true, 		% Dokumenttitel statt Dateiname anzeigen.
    pdflang={\sprache}, 			% Sprache des Dokuments.
]{hyperref}

% (Farb-)einstellungen für die Links im PDF
\hypersetup{%
    colorlinks=true, 		% Aktivieren von farbigen Links im Dokument
    linkcolor=black, 	    % Farbe festlegen
    citecolor=LinkColor,
    filecolor=LinkColor,
    menucolor=LinkColor,
    urlcolor=LinkColor,
    %linktocpage=true, 		% Nicht der Text sondern die Seitenzahlen in Verzeichnissen klickbar
    linktoc=all,            % Seitenzahlen und Text klickbar
    bookmarksnumbered=true 	% Überschriftsnummerierung im PDF Inhalt anzeigen.
}
% Workaround um Fehler in Hyperref, muss hier stehen bleiben
\usepackage{bookmark} %nur ein latex-Durchlauf für die Aktualisierung von Verzeichnissen nötig

% Schriftart in Captions etwas kleiner
\addtokomafont{caption}{\small}

% Literaturverweise (sowohl deutsch als auch englisch)
\iflang{de}{%
\usepackage[
    backend=bibtex,		% empfohlen. Falls biber Probleme macht: bibtex
    bibwarn=true,
    bibencoding=utf8,	% wenn .bib in utf8, sonst ascii
    sortlocale=de_DE,
    style=\zitierstil,
]{biblatex}
}
\iflang{en}{%
\usepackage[
    backend=bibtex,		% empfohlen. Falls biber Probleme macht: bibtex
    bibwarn=true,
    bibencoding=utf8,	% wenn .bib in utf8, sonst ascii
    sortlocale=en_US,
    style=\zitierstil,
]{biblatex}
}
\ladeliteratur{}

% Glossar
\usepackage[nonumberlist,toc]{glossaries}
\usepackage{blindtext} % Blindtext-Package. Common Usage: \blindtext für einzelnen Abschnitt, \Blindtext für mehrere Abschnitte

%%%%%% Additional settings %%%%%%

% Hurenkinder und Schusterjungen verhindern
% http://projekte.dante.de/DanteFAQ/Silbentrennung
\clubpenalty = 10000 % schließt Schusterjungen aus (Seitenumbruch nach der ersten Zeile eines neuen Absatzes)
\widowpenalty = 10000 % schließt Hurenkinder aus (die letzte Zeile eines Absatzes steht auf einer neuen Seite)
\displaywidowpenalty=10000

\setcounter{biburlnumpenalty}{100}
\setcounter{biburlucpenalty}{100}
\setcounter{biburllcpenalty}{100}

% Bildpfad
\graphicspath{{images/}}

% Einige häufig verwendete Sprachen
\lstloadlanguages{PHP, Python, Java, C, C++, bash}
\listingsettings{}
% Umbennung des Listings
\renewcommand\lstlistingname{\langlistingname}
\renewcommand\lstlistlistingname{\langlistlistingname}
\def\lstlistingautorefname{\langlistingautorefname}

% Abstände in Tabellen
\setlength{\tabcolsep}{\spaltenabstand}
\renewcommand{\arraystretch}{\zeilenabstand}

\usepackage{xspace}
\newcommand{\lastcontentpage}{}
\usepackage{amsfonts}

\usetikzlibrary{shapes,arrows,calc,positioning}
% !TeX root = ../dokumentation.tex

%% Some custom TikZ Styles for common UML Elements
%% This could be used instead of TikZ-UML

\tikzstyle{transition} = [%
    thin,
    -angle 45,
    rounded corners
]

\tikzstyle{rect} = [%
    draw,
    rounded corners,
]

\tikzstyle{singlestate} = [%
    draw,
    rounded corners,
    execute at begin node={\begin{varwidth}{10em}}, % Maximum Width
    execute at end node={\end{varwidth}},
    minimum width=10em                              % Minimum Width
]

\tikzstyle{synchronize} = [%
    draw,
    fill=black,
    minimum width=3cm
]

\tikzstyle{decision} = [%
    draw,
    diamond,
    minimum height=0.5cm,
    aspect=1
]

\tikzstyle{tnode} = [%
    font=\footnotesize
]

\tikzstyle{start} = [%
    circle,
    draw,
    fill=black
]

\tikzstyle{final} = [%
    circle,
    double,
    fill=black,
    double distance=0.1cm,
    draw
]

\usepackage{relsize}

\usepackage{censor}

\usepackage{eso-pic}

\usepackage{adjustbox}

%% Paket um Textteile drehen zu können
% \usepackage{rotating}
%% Paket um Seite im Querformat anzuzeigen
%\usepackage{lscape}

\newcommand\Watermark{%
    \put(0,0){%
        \parbox[b][\paperheight]{\paperwidth}{%
            \vfill
            \includepdf[scale=0.8,angle=50,pages={1},pagecommand={}]{ads/watermark}
            \vfill
        }
    }
}

\ifrefWithPages
    %RJG8FE: add a pageref to autoref whenever the referenced page is not the same as the current one
    %        useful for printed documents without clickable hyperlinks
    \AtBeginDocument{\let\oldautoref\autoref}
    \AtBeginDocument{
        \renewcommand{\autoref}[1]{%
            \oldautoref{#1}%
            \ifthenelse{\thepage=\pageref{#1}}% if current page number equals the referenced page number
            {}% then add nothing
            { (S. \pageref{#1})}% else add the text
        }
    }
\fi

\usepackage{amssymb} % Erweiterung der Symbole in Mathematikumgebung

\iflang{de}{\usepackage{icomma}} % Europäsiches Komma in Formeln

\usepackage{tabularx}
\usepackage{tabulary}

%%%%%% Custom Enviroments %%%%%%
\newenvironment{conditions}[1][\formelbeschreibung]
  {%
      #1\tabularx{\textwidth-\widthof{#1}}[t]{
      >{$}l<{$}  @{} >{${}}c<{{}$} @{} X@{}
      }%
  }
  {\endtabularx\\[\belowdisplayskip]}
%%%%%% Custom Commands %%%%%%
\newrobustcmd*{\citecompleteauthor}{\AtNextCite{\DeclareNameAlias{labelname}{given-family}}\citeauthor}